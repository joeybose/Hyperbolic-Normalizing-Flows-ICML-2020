\section{Background on Hyperbolic Geometry}
We briefly review the Lorentz model of hyperbolic geometry. We are not
assuming background in Riemannian geometry (Appendix \ref{riemannian_geometry_appendix}); \cut{one can understand concepts
like curvature intuitively without needing a long digression into
Riemannian geometry.} a very useful source is \cite{Ratcliffe94}. Henceforth, for notational clarity, we use boldface font to denote points on the hyperboloid manifold.

\subsection{Lorentz Model of Hyperbolic Geometry}
Hyperbolic space is a manifold with constant negative curvature.  Historically it arose as an example of a geometry that is internally consistent but breaks Euclid's parallel postulate.  An $n-$dimensional hyperbolic space, $\mathbb{H}^n_K$, is the unique, complete, simply-connected $n-$dimensional Riemannian manifold of constant negative curvature, $K$.  There are multiple models of Hyperbolic space but for our purposes the Lorentz model is the most convenient, since it is equipped with relatively simple explicit formulas and useful numerical stability properties \cite{nickel2018learning}.

The Lorentz model embeds hyperbolic space $\mathbb{H}^n_K$ within the $n+1-$dimensional Minkowski space, defined as the manifold $\mathbb{R}^{n+1}$ equipped with the following inner product:
\begin{equation}\label{eq:lorentzmetric}
    \langle \textbf{x}, \textbf{y} \rangle_{\mathcal{L}} := -x_0y_0 + x_1y_1 + \dots + x_ny_n,
\end{equation}
which has the type 
$\langle \cdot, \cdot \rangle_{\mathcal{L}}: \mathbb{R}^{n+1} \times \mathbb{R}^{n+1} \to \mathbb{R}$.
%It is indefinite, which may come as a shock to those used to the traditional definition of inner product from linear algebra, but this is a well-defined notion extensively used in special relativity for over a century.  
It is common to denote this space as $\mathbb{R}^{1,n}$ to emphasize the distinct role of the zeroth coordinate.
In the Lorentz model, we model hyperbolic space as the (upper sheet of) the hyperboloid embedded in Minkowski space. It is a remarkable fact that though the Lorentzian metric (Eq.~\ref{eq:lorentzmetric}) is indefinite,
the induced Riemannian metric $g_{\mb{x}}$ on the unit hyperboloid is positive definite \cite{Ratcliffe94}. The $n-$Hyperbolic space with constant negative curvature $K$ with origin $\textbf{o} = (1/K, 0, \dots, 0)$, is a Riemannian manifold $(\mathbb{H}^{n}_K,g_{\mb{x}})$ where
\begin{equation*}
    \mathbb{H}^{n}_K := \{x \in \mathbb{R}^{n+1}:  \langle \textbf{x}, \textbf{x} \rangle_{\mathcal{L}} = 1/K, \ x_0 > 0, \ K<0 \},
\end{equation*}

%If we define $d_{\mathbb{H}^{n}_K} = $
%Indeed, if we define $d_{\mathbb{H}^{n}_K}(\mathbf{x},\mathbf{y})$ by $\langle \textbf{x},\textbf{y} \rangle_{\mathcal{L}}  = ||\bx||||\by||\cosh d_{\mathbb{H}^{n}_K}(\bx,\by)$ then $d_{\mathbb{H}^{n}_K}$ is a metric on the hyperbolic space \cite{Ratcliffe94}.
%; see~\cite{Ratcliffe94} page 64.
%Describing hyperbolic space in this way is convenient as one can use the coordinates in the embedding space to write explicit formulas for derivatives and other operations.  
%When convenient we shall also use the generalized radius $R = 1/\sqrt{-K}$ in place of the curvature.

Equipped with this, the induced distance between two points $(\mb{x},\mb{y})$ in $\mathbb{H}^{n}_K$ is given by
\begin{equation}
    d(\textbf{x},\textbf{y})_{\mathcal{L}} := \frac{1}{\sqrt{-K}}\arccosh(-K \langle \textbf{x}, \textbf{y} \rangle_{\mathcal{L}}).
\end{equation}
%This is obtained by inverting the relation for $\eta$ described above.

The tangent space to the hyperboloid at the point $\textbf{p}$ can also be described as an embedded subspace of $\mathbb{R}^{1,n}$.  
It is given by the set of points satisfying the orthogonality relation with respect to the Minkowski inner product \footnote{It is also known as the Lorentz inner product in some texts.},
\begin{equation}
    \mathcal{T}_{\textbf{p}}\mathbb{H}^n_K := \{u: \langle u, \textbf{p} \rangle_{\mathcal{L}} = 0 \}.
\end{equation}
Of special interest are vectors in the tangent space at the origin of $\mathbb{H}^{n}_K$ whose norm under the Minkowski inner product equivalent to the conventional Euclidean norm. That is $v \in \mathcal{T}_{\textbf{o}}\mathbb{H}^{n}_K$ is a vector such that $v_0 = 0$ and $||\textbf{v}||_{\mathcal{L}} := \sqrt{\langle \textbf{v}, \textbf{v}
\rangle_{\mathcal{L}}} = ||\textbf{v}||_2$.  Thus \emph{at the origin} the
partial derivatives with respect to the ambient coordinates, $\mathbb{R}^{n+1}$, defines the
covariant derivative.  
Of course, this is not generally true at other points but we will define our transformations such that we only require derivatives at the origin.

\xhdr{Projections}
Starting from the extrinsic view by which we mean $\mathbb{H}^{n}_K \subset \mathbb{R}^{n+1}$, we may project any vector $x \in \mathbb{R}^{n+1}$ on to the hyperboloid using the shortest Euclidean distance:
\begin{equation}
    \textnormal{proj}_{\mathbb{H}^{n}_K(x)} = \frac{x}{\sqrt{-K}||x||_{\mathcal{L}}}
\end{equation}
Furthermore, by definition a point on the hyperboloid satisfies $\langle \textbf{x}, \textbf{x} \rangle_{\mathcal{L}} = 1/K$ and thus when provided with $n$ coordinates $\hat{x} = (x_1, \dots, x_{n})$ we can always determine the missing coordinate to get a point on $\mathbb{H}^n_K$,
\begin{equation}
    \label{eqn:hyperboloid_projection}
    x_0 = \sqrt{||\hat{x}||^2_2 + \frac{1}{K}}
\end{equation}
We later exploit this fact in \ref{wrapped_hyerboloid_coupling_section}, to determine points on $\mathcal{H}^n_K$ that our normalizing flow should tranport to.

\xhdr{Exponential Map}
The exponential map takes a vector, $v$, in the tangent space of a point $\textbf{x} \in \mathbb{H}^{n}_K$ to a point on the manifold---i.e., $\textbf{y} = \textnormal{exp}^K_\textbf{x}(v): \mathcal{T}_{\textbf{x}} \mathbb{H}^{n}_K \to \mathbb{H}^{n}_K$ by moving a unit length along the \textit{geodesic}, $\gamma$ (straightest parametric curve), uniquely defined by $\gamma(0) = \textbf{x}$ with direction $\gamma '(0)= v$. The closed form expression for the Lorentz model is then given by
\begin{equation}
    \textnormal{exp}^K_{\textbf{x}} (v) = \cosh \Big(\frac{||v||_{\mathcal{L}}}{R} \Big)\textbf{x} +  \sinh \Big(\frac{||v||_{\mathcal{L}}}{R} \Big)\frac{Rv}{||{v}||_{\mathcal{L}}},
\end{equation}
where we used the generalized radius $R = 1/\sqrt{-K}$ in place of the curvature.

\xhdr{Logarithmic Map}
As the inverse of the exponential map, the logarithmic map takes a point, $\textbf{y}$, on the manifold back to the tangent space of another point $\textbf{x}$ also on the manifold. In the Lorentz model this is defined as
\begin{equation}
    \log^K_{\textbf{x}}{\textbf{y}} = \frac{\arccosh(\alpha)}{\sqrt{\alpha^2 - 1}}(\textbf{y} - \alpha \textbf{x}),
\end{equation}
where $\alpha = K\langle \textbf{x}, \textbf{y} \rangle_{\mathcal{L}}$.

\xhdr{Parallel Transport}
The parallel transport for two points $\textbf{x},\textbf{y} \in \mathbb{H}^{n}_K$ is a map that carries the vectors in $v \in \mathcal{T}_{\textbf{x}}\mathbb{H}^{n}_K$ to corresponding vectors at $v' \in \mathcal{T}_{\textbf{y}}\mathbb{H}^{n}_K$ along the geodesic. That is vectors are connected between the two tangent spaces such that the covariant derivative is unchanged. For Riemannian Manifolds we use the Levi-Civita connection. Parallel transport is a map that preserves the metric tensor, \ie, $\langle \textnormal{PT}^K_{\textbf{x} \to \textbf{y}}(v), \textnormal{PT}^K_{\textbf{x} \to \textbf{y}}(v') \rangle_{\mathcal{L}} = \langle v, v' \rangle_{\mathcal{L}}$ and in the Lorentz model is given by
\begin{align}
    \textnormal{PT}^K_{\textbf{x} \to \textbf{y}}(v) & = v - \frac{ \langle \log^K_{\textbf{x}}(\textbf{y}), v \rangle_{\mathcal{L}}}{d(\textbf{x},\textbf{y})_{\mathcal{L}}} (\log^K_{\textbf{x}}(\textbf{y})+ \log^K_{\textbf{y}}(\textbf{x})) \nonumber \\
    & = v + \frac{ \langle \textbf{y}, v \rangle_{\mathcal{L}}}{R^2 - \langle \textbf{x}, \textbf{y} \rangle_{\mathcal{L}}} (\textbf{x}+ \textbf{y}),
\end{align}
where $\alpha$ is as defined above. Another useful property is that the inverse parallel transport simply carries the vectors back along the geodesic and is simply

\begin{equation}
      (\textnormal{PT}^{K}_{\textbf{x} \to \textbf{y}}(v))^{-1} = \textnormal{PT}^K_{\textbf{y} \to \textbf{x}}(v).
\end{equation}

\subsection{Probability Distributions on Hyperbolic Spaces}

Probability distributions can be defined on Riemannian manifolds which includes $\mathbb{H}^n_K$ as a special cases, by relating an infinitesimal volume on the manifold to a corresponding volume in a homeomorphic Euclidean space by the conservation of measure.
In particular, given the Riemannian manifold $\mathcal{M}(\textbf{z})$ and its metric $g_\mb{z}$, we have $\int p(\textbf{z}) d\mathcal{M} (\textbf{z}) = \int p(\textbf{z})\sqrt{|g_\mb{z}|}d \textbf{z}$, where $d \textbf{z}$ is the Lebesgue measure. We now briefly survey three distinct generalizations of the normal distribution to Riemannian manifolds.\cut{ following ~\cite{mathieu2019continuous,skopek2019mixed}.}

\xhdr{Riemannian Normal}
The first is the Riemannian normal~\cite{pennec2006intrinsic, said2014new}, which is derived from maximizing the entropy given a mean $\mu$ and a variance $\sigma$.
Specifically, we have $\mathcal{N}_{\mathcal{M}}(\textbf{z} \vert \mu, \sigma^{2}) = \frac{1}{Z} \exp \left( - d_{\mathcal{M}}(\mu, \textbf{z})^2 / 2 \sigma^{2} \right)$, where $d_{\mathcal{M}}$ is the induced distance and $Z$ is the normalization constant \cite{said2014new, mathieu2019continuous}.

\xhdr{Restricted Normal}
One can also restrict sampled points from the normal distribution in the ambient space to the manifold.
One example is the Von Mises distribution on the unit circle and its generalized version, \ie, Von Mises-Fisher distribution on the hypersphere~\cite{davidson2018hyperspherical}.

\xhdr{Wrapped Normal}
Finally, we can define a wrapped normal distribution~\cite{falorsi2019reparameterizing,nagano2019wrapped}, which is obtained by (1) sampling a point $v$ from a normal distribution defined in the tangent space at the origin $\textbf{o}$; (2) parallel transporting the sample $v$ from the tangent space at $\textbf{o}$ to the tangent space of another point $\boldsymbol{\mu}$ on the manifold to obtain $u$; (3) mapping $u$ from the tangent space to the manifold using the exponential map at $\boldsymbol{\mu}$.
Sampling from such a distribution is straightforward and the probability density can be obtained via the change of variable formula,
\begin{align}
    \log p(\textbf{z}) = \log p(v) - (n-1) \log \left(\frac{\sinh{ ( \Vert u \Vert_{\mathcal{L}} ) }}{ \Vert u \Vert_{\mathcal{L}} } \right),
\end{align}
where $p(\textbf{z})$ is the wrapped normal distribution and $p(v)$ is the normal distribution in the tangent space of $\textbf{o}$.




