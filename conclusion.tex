\section{Conclusion}
In this paper we introduce two novel normalizing flows on hyperbolic spaces. Our first coupling transform, $\mathcal{T}C$, lifts normalizing flow operations to the tangent space at the origin. We build upon $\mathcal{T}C$ to design the $\mathcal{W}\mathbb{H}C$ layer which exploits disparate regions of the manifold. We further show that our flows are efficient to sample from, easy to invert and require only $\mathcal{O}(n)$ cost to compute the change in volume. We demonstrate the effectiveness of constructing hyperbolic normalizing flows for latent variable modelling of hierarchical data. We empirically observe improvements in structured density estimation, graph reconstruction and also generative modelling of tree structured data withe large qualitative improvements in generated sample quality compared to Euclidean methods. One important limitation is in the numerical error introduced by clamping operations which prevent the creation of deep flow architectures. We hypothesize that this is an inherent limitation of the Lorentz model which maybe alleviated with newer isometries of hyperbolic geometry that use integer-based tiling \cite{yu2019numerically}. While we considered hyperbolic generalizations of the coupling transforms to define our normalizing flows they represent only one class of possible invertible transformations. Designing new classes of invertible transformation like autoregressive and residual flows but for spaces of constant curvature is an interesting direction for future work.