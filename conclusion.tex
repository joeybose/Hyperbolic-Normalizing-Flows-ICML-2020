\section{Conclusion}
In this paper we introduce two concrete instantiations of normalizing flows on hyperbolic spaces by defining novel coupling transforms. Our first coupling transform, $\mathcal{T}C$, lifts normalizing flow operations by extensively using the tangent space at the origin. We also introduce the $\mathcal{W}\mathbb{H}C$ layer which is specifically designed to explicitly use different regions of the manifold. We further prove that our flows are efficient to sample from, easy to invert and requires only $\mathcal{O}(n)$ cost to compute the change in volume. We demonstrate the effectiveness of constructing normalizing flows for latent variable modelling of hierarchical data where we observe notable gains in structured density estimation compared to Euclidean methods in low dimensions. We further evaluate our approach in graph reconstruction tasks where we see moderate improvements. Lastly, we show generative modelling capabilities of hyperbolic normalizing flows on tree structured data where we observe large qualitative improvements in generated sample quality. 

In terms of limitations and directions for future work, one important limitation is in the numerical error introduced by clamping operations which prevent the creation of deep flow architectures. We hypothesize that this is an inherent limitation of the Lorentz model which maybe alleviated with newer isometries of hyperbolic geometry that use integer-based tiling. While we considered hyperbolic generalizations of the coupling transforms to define our normalizing flows they represent only one class of possible invertible transformations. Designing new classes of invertible transformation like autoregressive and residual flows but for spaces of constant curvature is an interesting direction for future work.